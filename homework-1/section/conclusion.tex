\newpage
\section{Conclusions and Future Work}
\label{sec:conclusion}
 
In this work, we presented our approach to the \ac{CLEF} \textit{Long Eval LAB 2022} task, which aimed to develop an effective and efficient search engine for web documents. \\
Our approach consisted of using a combination of different techniques, including query expansion, re-ranking, and the use of large language models such as \textit{ChatGPT} and \textit{SBERT}. \\
Our experiments showed that our approach achieved good results in terms of effectiveness and efficiency, outperforming the baseline system provided by \ac{CLEF}. Specifically, we found that combining two different scores in the re-ranking phase led to significant improvements in the retrieval performance. Moreover, we identified several areas for future work that could further improve the effectiveness and efficiency of our approach. \\
One possible direction for future work is to find better ways to combine scores or add other scores to the re-ranking phase. We plan to explore different combinations of scores and investigate the use of other large language models, such as other available \textit{BERT} models trained, or to train some specifically for this task. \\
Another area for future work is to find better prompts \cite{wang2023chatgpt} to use in \textit{ChatGPT} for improving query expansion. We also plan to investigate the use of other \ac{LLM} techniques for query expansion. \\
We also want to explore ways to increase the similarity in \textit{SBERT}~\cite{reimers-2019-sentence-bert}, to increase the number of relevant documents found in the re-ranking phase. One possible approach is to fine-tune the \textit{SBERT}~\cite{reimers-2019-sentence-bert} model on our specific task. \\
Another direction for future work is to index documents as vectors and use them directly, instead of calculating them in re-ranking. This trade-off would result in the loss of one of the scores, but it would increase the re-ranking speed. \\
Finally, we plan to use links inside documents to extract details that may improve the searching results. We may try to find keywords in the URL path and use them to find their domain authority and take this aspect into account in the score computation.
