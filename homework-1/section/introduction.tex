\enlargethispage{2\baselineskip}
\section{Introduction}
\label{sec:introduction}

Recent research has shown that the performance of information retrieval systems can deteriorate over time as the data they are trained on becomes less relevant to current search queries. 
This problem is particularly acute when dealing with temporal information, as web documents and user search preferences evolve over time. 
In this paper, we propose a solution to this problem by developing an information retrieval system that can adapt to changes in the data over time, while maintaining high performance.

Our approach involves using the training data provided by \textit{Qwant}~\cite{qwant} search engine, which includes user searches and web documents in both French and English.
We believe that this data will enable our system to better adapt to changes in user search behavior and the content of web documents.

The remainder of this paper is organized as follows: 
Section~\ref{sec:methodology} describes our approach in more detail, including the different techniques we used. 
Section~\ref{sec:setup} explains our experimental setup, including the datasets and evaluation metrics we used. 
Section~\ref{sec:results} presents our main findings and analyzes them based on the choices we made regarding various techniques. 
Finally, Section~\ref{sec:conclusion} summarizes our conclusions and outlines potential avenues for future work.